% Options for packages loaded elsewhere
\PassOptionsToPackage{unicode}{hyperref}
\PassOptionsToPackage{hyphens}{url}
\PassOptionsToPackage{dvipsnames,svgnames,x11names}{xcolor}
%
\documentclass[
  letterpaper,
  DIV=11,
  numbers=noendperiod]{scrartcl}
\usepackage[margin=0.75in]{geometry}

\usepackage{amsmath,amssymb}
\usepackage{iftex}
\ifPDFTeX
  \usepackage[T1]{fontenc}
  \usepackage[utf8]{inputenc}
  \usepackage{textcomp} % provide euro and other symbols
\else % if luatex or xetex
  \usepackage{unicode-math}
  \defaultfontfeatures{Scale=MatchLowercase}
  \defaultfontfeatures[\rmfamily]{Ligatures=TeX,Scale=1}
\fi
\usepackage{lmodern}
\ifPDFTeX\else  
    % xetex/luatex font selection
\fi
% Use upquote if available, for straight quotes in verbatim environments
\IfFileExists{upquote.sty}{\usepackage{upquote}}{}
\IfFileExists{microtype.sty}{% use microtype if available
  \usepackage[]{microtype}
  \UseMicrotypeSet[protrusion]{basicmath} % disable protrusion for tt fonts
}{}
\makeatletter
\@ifundefined{KOMAClassName}{% if non-KOMA class
  \IfFileExists{parskip.sty}{%
    \usepackage{parskip}
  }{% else
    \setlength{\parindent}{0pt}
    \setlength{\parskip}{6pt plus 2pt minus 1pt}}
}{% if KOMA class
  \KOMAoptions{parskip=half}}
\makeatother
\usepackage{xcolor}
\setlength{\emergencystretch}{3em} % prevent overfull lines
\setcounter{secnumdepth}{-\maxdimen} % remove section numbering
% Make \paragraph and \subparagraph free-standing
\makeatletter
\ifx\paragraph\undefined\else
  \let\oldparagraph\paragraph
  \renewcommand{\paragraph}{
    \@ifstar
      \xxxParagraphStar
      \xxxParagraphNoStar
  }
  \newcommand{\xxxParagraphStar}[1]{\oldparagraph*{#1}\mbox{}}
  \newcommand{\xxxParagraphNoStar}[1]{\oldparagraph{#1}\mbox{}}
\fi
\ifx\subparagraph\undefined\else
  \let\oldsubparagraph\subparagraph
  \renewcommand{\subparagraph}{
    \@ifstar
      \xxxSubParagraphStar
      \xxxSubParagraphNoStar
  }
  \newcommand{\xxxSubParagraphStar}[1]{\oldsubparagraph*{#1}\mbox{}}
  \newcommand{\xxxSubParagraphNoStar}[1]{\oldsubparagraph{#1}\mbox{}}
\fi
\makeatother


\providecommand{\tightlist}{%
  \setlength{\itemsep}{0pt}\setlength{\parskip}{0pt}}\usepackage{longtable,booktabs,array}
\usepackage{calc} % for calculating minipage widths
% Correct order of tables after \paragraph or \subparagraph
\usepackage{etoolbox}
\makeatletter
\patchcmd\longtable{\par}{\if@noskipsec\mbox{}\fi\par}{}{}
\makeatother
% Allow footnotes in longtable head/foot
\IfFileExists{footnotehyper.sty}{\usepackage{footnotehyper}}{\usepackage{footnote}}
\makesavenoteenv{longtable}
\usepackage{graphicx}
\makeatletter
\newsavebox\pandoc@box
\newcommand*\pandocbounded[1]{% scales image to fit in text height/width
  \sbox\pandoc@box{#1}%
  \Gscale@div\@tempa{\textheight}{\dimexpr\ht\pandoc@box+\dp\pandoc@box\relax}%
  \Gscale@div\@tempb{\linewidth}{\wd\pandoc@box}%
  \ifdim\@tempb\p@<\@tempa\p@\let\@tempa\@tempb\fi% select the smaller of both
  \ifdim\@tempa\p@<\p@\scalebox{\@tempa}{\usebox\pandoc@box}%
  \else\usebox{\pandoc@box}%
  \fi%
}
% Set default figure placement to htbp
\def\fps@figure{htbp}
\makeatother

\KOMAoption{captions}{tableheading}
\makeatletter
\@ifpackageloaded{caption}{}{\usepackage{caption}}
\AtBeginDocument{%
\ifdefined\contentsname
  \renewcommand*\contentsname{Table of contents}
\else
  \newcommand\contentsname{Table of contents}
\fi
\ifdefined\listfigurename
  \renewcommand*\listfigurename{List of Figures}
\else
  \newcommand\listfigurename{List of Figures}
\fi
\ifdefined\listtablename
  \renewcommand*\listtablename{List of Tables}
\else
  \newcommand\listtablename{List of Tables}
\fi
\ifdefined\figurename
  \renewcommand*\figurename{Figure}
\else
  \newcommand\figurename{Figure}
\fi
\ifdefined\tablename
  \renewcommand*\tablename{Table}
\else
  \newcommand\tablename{Table}
\fi
}
\@ifpackageloaded{float}{}{\usepackage{float}}
\floatstyle{ruled}
\@ifundefined{c@chapter}{\newfloat{codelisting}{h}{lop}}{\newfloat{codelisting}{h}{lop}[chapter]}
\floatname{codelisting}{Listing}
\newcommand*\listoflistings{\listof{codelisting}{List of Listings}}
\makeatother
\makeatletter
\makeatother
\makeatletter
\@ifpackageloaded{caption}{}{\usepackage{caption}}
\@ifpackageloaded{subcaption}{}{\usepackage{subcaption}}
\makeatother

\usepackage{bookmark}

\IfFileExists{xurl.sty}{\usepackage{xurl}}{} % add URL line breaks if available
\urlstyle{same} % disable monospaced font for URLs
\hypersetup{
  colorlinks=true,
  linkcolor={blue},
  filecolor={Maroon},
  citecolor={Blue},
  urlcolor={Blue},
  pdfcreator={LaTeX via pandoc}}


\author{}
\date{}

\begin{document}
\thispagestyle{empty}

\noindent
\begin{minipage}[c]{0.65\textwidth}
\section{BUSI 721: Data Driven Finance \\ Spring 2026}
\end{minipage}%
\hfill
\begin{minipage}[c]{0.3\textwidth}
\raggedleft
\includegraphics[height=0.5in,keepaspectratio]{RiceBusiness-transparent-logo-sm.png}
\end{minipage}

\vspace{0.5cm}

\subsection{Instructor}\label{instructor}

\href{http://kerryback.com}{Kerry Back} \\
J. Howard Creekmore Professor of
Finance and Professor of Economics \\
Rice University\\
kerry.e.back@rice.edu

\subsection{Course Description}\label{course-description}

This course is an introduction to investments, including the analysis of
corporate investment projects, and to derivative securities. It will run
on two tracks. The pre-recorded videos and weekly assignments are about
investment analysis. The weekly live sessions will be about derivative
securities. Grading will be equally weighted between investments
assignement and a final exam on derivative securities.

\paragraph{Investments}\label{investments}

We begin with foundational issues such as retirement planning and
mortgage calculations. The next part of the course describes markets,
assets, and properties of returns. We then study how to construct
efficient portfolios of assets. The final part of the course pertains to
the financial analysis of corporate projects.

Assignments and videos are linked on the course Canvas site. Copies of
the slides are at
\href{https://busi721-901.kerryback.com/slides.html}{busi721-901.kerryback.com/slides.html}.
Jupyter notebooks containing the code in the slides are at
\href{https://github.com/kerryback/2023-721-901-binder}{github.com/kerryback/2023-721-901-binder}.
As the course progresses, solutions to the assignments will be posted at
\href{https://github.com/kerryback/busi721-901/tree/main/solutions}{github.com/kerryback/busi721-901/tree/main/solutions}.
A good reference for the course material is this free textbook by Ivo
Welch of UCLA: 
\href{https://corpfin.ivo-welch.info/read/}{corpfin.ivo-welch.info/read/}.

\paragraph{Derivative Securities}\label{derivative-securities}

We will follow the free online textbook written by your instructor in
conjunction with Hong Liu of Washington University in St.~Louis and Mark
Loewenstein of the University of Maryland:
\href{https://book.derivative-securities.org}{book.derivative-securities.org}.
The book provides a link to a NotebookLM notebook that you can query
regarding the book (NotebookLM is a RAG app provided free of charge by
Google and powered by Google Gemini). It also provides a link to Jupyter
notebooks containing all the code in the book. The lecture slides that
we will cover in our weekly live sessions are (will be) posted at
\href{https://slides.derivative-securities.org}{slides.derivative-securities.org}.

More information about the final exam will be provided later.




\end{document}

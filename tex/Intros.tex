\documentclass[english,12pt]{amsart}
\RequirePackage{geometry,amsmath,graphicx,babel}

%%%%%%%%%%%%%%%%%%%%%%%%%%%%%%%%%%%%%%%%%%%%%%%%%%%%%%%%%%%%%%%%%%%%%%%%%%%%%%%%%%%%%%%%%%

\geometry{verbose,letterpaper,tmargin=1in,bmargin=1in,lmargin=1.25in,rmargin=1.25in,headheight=0.5in,footskip=0.5in}

\setlength{\parskip}{\bigskipamount}

\setlength{\parindent}{0pt}


\usepackage{hyperref}

%%%%%%%%%%%%%%%%%%%%%%%%%%%%%%%%%%%%%%%%%%%%%%%%%%%%%%%%%%%%%%%%%%%%%%%%%%%%

\begin{document}

\begin{center}
    Module 1\\
   \end{center}



How much do you need to save each year to provide for a comfortable retirement?  How do you adjust your savings plan for inflation?  What would the mortgage payments be on that house you want to buy?  Given what you can afford to pay each month, how much can you afford to borrow?  We'll answer these questions and more in this module.  We'll track retirement savings accounts or loan accounts and see how to calculate future and present values.  We'll cover the interest-rate math and also introduce code to perform the calculations.

\vskip \baselineskip\hrule\vskip \baselineskip

\begin{center}
    Module 2\\
   \end{center}


Are stock returns normally distributed?  Does today's return predict tomorrow's return or this year's return predict next year's return?  How is it possible for investors in a stock to lose money over time when the average return of the stock is positive?  Is the stock market risky for long-run investors?  What were the best 20 years and what were the worst 20 years in the U.S. market?  These questions are the topic of this module.  We'll grab stock price data from the web and calculate and analyze returns.  We'll also run simulations to look at risk in the long run.

\vskip \baselineskip\hrule\vskip \baselineskip

\begin{center}
    Module 3\\
  \end{center}


We're going to discuss how markets operate and how investors trade.  We'll discuss different types of orders and how your order is handled once you submit it.  We'll see why borrowing money to invest is called using leverage, and we'll see how and why we might sell things in financial markets that we don't own.  We'll discuss how mutual funds work and how exchange traded funds work, and why exchange traded funds have become increasingly popular.  We'll pay special attention to bond markets, discussing how bonds are issued and traded and why bond prices fall when interest rates rise.

\vskip \baselineskip\hrule\vskip \baselineskip

\begin{center}
    Module 4\\
\end{center}

Don't put all your eggs in one basket is a good maxim for investing.  Risk is reduced by diversification.  How much it is reduced and what the optimal mix of assets is are very much influenced by the correlations and covariances of asset returns.  We'll see why the correct measure of risk for an investor is how much an asset covaries with the investor's portfolio return, and we'll learn how to calculate the least risk portfolio achieving any target expected return.  This leads to what is called the Capital Asset Pricing Model, which is a model used by companies to determine the hurdle rate for corporate projects.




\vskip \baselineskip\hrule\newpage

\begin{center}
    Module 5\\
  \end{center}

  We'll introduce a simple parameterization of risk aversion and see how to compute the optimal portfolio for a given level of risk aversion.  We'll find least-risk and optimal portfolios with different borrowing and saving rates and with limitations on short selling.  We'll also discuss the reliability or lack thereof of estimates of expected returns and risks based solely on sample data.  Finally, we'll apply everything we've learned about portfolios to look at portfolios of stocks, bonds, and gold.  

  \vskip \baselineskip\hrule\vskip \baselineskip

\begin{center}
    Module 6\\
\end{center}

Companies create value for shareholders by making investments that have positive net present values.  The net present value or NPV is the present value of the project cash flows.  To forecast cash flows, we forecast a balance sheet for the project consisting of long-term assets adjusted for depreciation and working capital.  We also forecast an income statement, and from net income and changes in the balance sheet we compute cash flows.  Economic value added or EVA in each year of a project's lifetime is equal to income less a charge for the capital tied up in the project.  The present value of the EVA equals the NPV.

\vskip \baselineskip\hrule\vskip \baselineskip

\begin{center}
    Module 7\\
\end{center}

We'll learn to calculate the rate of return of a corporate project called the internal rate of return.  We'll also develop a two-stage growth model for valuing companies.  The second stage of the model assumes growth of cash flows at a constant rate forever.  The required rate of return for a project or for valuation of a company is a weighted average of the cost of equity and the after-tax cost of debt.  Because interest payments are tax deductible, the weighted average cost of capital can be lowered and value created for shareholders by using more debt capital and less equity capital, up to a point.


\end{document}